%NO MODIFICAR ESTA SECCION!
\documentclass{article} % Define la clase del documento, en este caso, un artículo

\usepackage[letterpaper,margin=3cm]{geometry} % Configura el tamaño del papel y los márgenes del documento
\usepackage{graphicx} % Permite la inserción de imágenes
%\usepackage[spanish]{babel}% Activar esta configuración para informes en español, ajusta el idioma del documento
\usepackage[usenames]{color} % Permite el uso de colores definidos por nombre en el documento
\usepackage{hyperref} % Habilita enlaces y referencias dentro del documento
\hypersetup{colorlinks=true, linkcolor = black, citecolor= black} % Configura el color de los enlaces y citas
\usepackage{booktabs} % Proporciona comandos para crear tablas de alta calidad
\usepackage{natbib} % Permite el uso de citas y referencias bibliográficas con diferentes estilos
\usepackage{tikz} % Permite la creación de gráficos y diagramas vectoriales directamente en LaTeX
\usepackage{float} % Para controlar la posición de los elementos flotantes, como imágenes, con la opción [H]
\bibliographystyle{agsm} % Define el estilo de citas y bibliografía (en este caso, el estilo AGSM)
\usepackage{diagbox} % Permite crear celdas con líneas diagonales en tablas
\usepackage{listings} % Permite la inclusión y formateo de código fuente en el documento
\usepackage{xcolor} % Paquete para definir y usar colores en el documento
\usepackage{parskip} % Añade espacio entre párrafos en lugar de sangrías
\usepackage{fancyhdr} % Permite personalizar encabezados y pies de página
\usepackage{amsmath} % Proporciona una amplia variedad de entornos y comandos matemáticos

\pagestyle{fancy} % Usa el estilo fancyhdr
\fancyhf{} % Borra todos los encabezados y pies de página
\renewcommand{\headrulewidth}{0pt}
\renewcommand{\footrulewidth}{0pt} % Desactiva la línea horizontal predeterminada en el pie
\setlength{\headheight}{2cm} % Ajusta la altura del encabezado para hacer espacio para la línea
\fancyhead[L]{\raisebox{0.20cm}{\textbf{Fluid Mechanics}}} % Añade el texto en la parte izquierda del encabezado, subiéndolo ligeramente
\fancyhead[R]{\raisebox{0.1cm}{\includegraphics[width=0.25\linewidth]{LOGO_UNIVERSIDAD.jpg}}} % Añade la imagen en la parte derecha del encabezado y súbela un poco
\fancyhead[C]{\rule{\textwidth}{0.6pt}} % Añade una línea horizontal superior centrada
\fancyfoot[C]{\rule{\textwidth}{0.6pt}} % Añade una línea horizontal en el pie de página centrada
\fancyfoot[R]{\raisebox{-1.5\baselineskip}{\thepage}} % Coloca el número de página a la derecha, con suficiente espacio debajo de la línea
\geometry{top=3cm, bottom=2.5cm} % Ajusta los márgenes superior e inferior

% Definición de colores al estilo Visual Studio Code
\definecolor{codegreen}{rgb}{0.25,0.49,0.48} % Comentarios
\definecolor{codegray}{rgb}{0.5,0.5,0.5} % Números y anotaciones
\definecolor{codepurple}{rgb}{0.58,0,0.82} % Palabras clave
\definecolor{backcolour}{rgb}{0.95,0.95,0.92} % Color de fondo

% Configuración del estilo de las celdas de código
\lstset{
    backgroundcolor=\color{backcolour},   % color de fondo; necesita que el paquete color o xcolor esté cargado
    commentstyle=\color{codegreen},       % estilo de comentarios
    keywordstyle=\color{codepurple},      % estilo de palabras clave
    numberstyle=\tiny\color{codegray},    % estilo de los números de línea
    stringstyle=\color{red},              % estilo de las cadenas de texto
    basicstyle=\ttfamily\small,           % estilo del texto básico
    breakatwhitespace=false,              % ajustes de líneas sólo en espacios en blanco
    breaklines=true,                      % ajustar las líneas si son muy largas
    captionpos=b,                         % posición de la leyenda (abajo)
    keepspaces=true,                      % preserva los espacios en el texto; útil si se usa monoespaciado
    numbers=left,                         % dónde poner los números de línea
    numbersep=5pt,                        % qué tan lejos están los números de línea del código
    showspaces=false,                     % mostrar espacios con subrayados particulares; reemplaza 'showstringspaces'
    showstringspaces=false,               % subrayar los espacios dentro de las cadenas solo
    showtabs=false,                       % mostrar tabulaciones en el código con subrayados particulares
    tabsize=2,                            % tamaños de tabulación a 2 espacios
    language=TeX,                         % lenguaje del código
    morecomment=[l]\#,                    % reconocer # como inicio de comentario en Python
    frame=single,                         % agregar un marco simple alrededor del código
    rulecolor=\color{black}               % color del marco
}

\begin{document}
%----------------------------------------------------------------------------------------
%   PORTADA
%Modificar desde aqui en adelante
%----------------------------------------------------------------------------------------
\begin{titlepage}%Inicio de la carátula, solo modificar los datos necesarios
\newcommand{\HRule}{\rule{\linewidth}{0.5mm}} 
\center 
%----------------------------------------------------------------------------------------
%	ENCABEZADO
%----------------------------------------------------------------------------------------
\includegraphics[width=10cm]{LOGO_UNIVERSIDAD.jpg}\\ % Si esta plantilla se copio correctamente, va a llevar la imagen del logo de la facultad.OBS: Es necesario incluir el paquete: graphicx
\vspace{3cm}
%----------------------------------------------------------------------------------------
%	SECCION DEL TITULO
%----------------------------------------------------------------------------------------
\HRule \\[0.4cm]
{ \huge \bfseries Particle Image Velocimetry Laboratory}\\[0.4cm] % Titulo del documento
{ \huge \bfseries Fluid Mechanics}\\[0.4cm] % Titulo del documento
\HRule \\[1.5cm]
 \vspace{5cm}
%----------------------------------------------------------------------------------------
%	SECCION DEL AUTOR
%----------------------------------------------------------------------------------------
\begin{flushright}
    { \textbf{Professors:}\\
    Patricio Moreno\\
    Sebastián Sepúlveda\\
    \vspace{0.2cm}
    \textbf{Assistant:}\\
    Lukas Wolff\\
    \vspace{0.4cm}
}
\end{flushright}
\vspace{1cm}
%----------------------------------------------------------------------------------------
%	SECCION DE LA FECHA
%----------------------------------------------------------------------------------------
{\large \textbf{\today}}\\[2cm] % El comando \today coloca la fecha del dia, y esto se actualiza con cada compilacion, en caso de querer tener una fecha estatica, reemplazar el \today por la fecha deseada
\end{titlepage}
%----------------------------------------------------------------------------------------
%  INDICE
%----------------------------------------------------------------------------------------
%\newpage
%\tableofcontents
%\thispagestyle{plain} % Deshabilita el encabezado en la página del índice
%\thispagestyle{empty} % Deshabilita el número de página en la página del índice
%\newpage

%Se puede agregar un indice de figuras si es nesesario
%\newpage
%\listoffigures 
%\thispagestyle{plain} % Deshabilita el encabezado en la página del índice %
%\thispagestyle{empty}
%\newpage
%----------------------------------------------------------------------------------------
%   ACÁ EMPIEZA EL INFORME
\setcounter{page}{1} % Reinicia el contador de páginas
%----------------------------------------------------------------------------------------
%This is the format to follow for section titles
\section{Introduction}

In the first laboratory experiment, CFD techniques were used to simulate the flow of water through two infinite parallel plates. The issue with computational simulations is that it is not possible to know if they are correctly calibrated unless they are verified with experimental results. So, how can we study a fluid like air or water if they are transparent?
\\ \\
Throughout history, various techniques have been developed to achieve this goal, starting with observation:
\\ \\
In 1918, Porsche engineers set out to win Le Mans with their new prototype, the Porsche 917, but things did not go as expected. The car turned out to be unstable and difficult to drive. It was observed that the rear part of the car did not get dirty, which caught the attention of the engineers, as this meant that the air was not passing correctly through that area. This led to the decision to cut off the rear part of the car, giving birth to the Porsche 917K, one of the most dominant cars in the history of motorsports. \href{https://automedia.revsinstitute.org/1971-porsche-917k}{More information}. There's also the wind tunnel, which began its development in 1871 when Francis Herbert Wenham, an aeronautical engineer, built a wind tunnel to study the behavior of airplane wings. \href{https://www.grc.nasa.gov/www/k-12/WindTunnel/history.html}{Wind tunnel history}.
\\ \\
Nowadays, techniques such as PIV (Particle Image Velocimetry) and PTV (Particle Tracking Velocimetry) are being implemented in the industry to study the behavior of fluids in different situations. 
This laboratory experiment aims to study an incompressible Newtonian fluid using the PIV technique, where data such as velocity profile and shear stress will be obtained.

\section{Development}

For this exercise, you must develop a Python code, which, based on .txt files, should be able to obtain the velocity profile and shear stress from the PIV data obtained in the laboratory.
This curve should be compared with a theoretical calculation, where you should base it on Poiseuille's Law, which allows identifying the velocity profile in a laminar flow through a tube.

\begin{equation}
    u(r) = u_{max}(1 - \frac{r^2}{R^2})
\end{equation}

\begin{equation}
    u_{max} = 2 \cdot u_{avg}
\end{equation}

If you need more information, consult the following \href{https://www.simscale.com/docs/validation-cases/hagen-poiseuille-flow/}{link}.
\\ \\
You should calculate the shear stress profile using the following formula:

\begin{equation}
    \tau = \mu \frac{du}{dy}
\end{equation}

Likewise, the theoretical shear profile should be calculated and compared.

\newpage
\section{Questions}

In addition to the development, you must answer the following questions throughout your report:
\\ \\
What is the difference between PIV and PTV techniques?
\\ \\
How can AI (Artificial Intelligence) be implemented in PIV or PTV techniques?
\\ \\
What advantages does PIV or PTV have over the wind tunnel? Why is the wind tunnel still used in the industry?
\\ \\
Besides the techniques mentioned in this guide, what other techniques are used to study fluid behavior?
\\ \\
Why would be your argument to explain de differences between theoretical and experimental results?
\\ \\
What would be expected if the fluid velocity is increased? Conversely, what would be expected if the fluid viscosity is increased?
\\ \\
Why in the observed case, does the shear stress only have one component?
\\ \\
Mention the assumptions of the Poiseuille equation to derive the three given equations.


\section{Recommended Readings}

\href{https://github.com/LukasWolff2002/Lab_2_FM/blob/main/LECTURAS/Grant_1997.pdf}{Grant 1997}
\\ \\
\href{https://github.com/LukasWolff2002/Lab_2_FM/blob/main/LECTURAS/Hammad_1999.pdf}{Hammad 1999}
\\ \\
\href{https://github.com/LukasWolff2002/Lab_2_FM/blob/main/LECTURAS/Nezu_2011.pdf}{Nezu 2011}

\section{Data}

The txt files with the necessary information to run the codes are available at the following \href{https://github.com/LukasWolff2002/Lab_2_FM/tree/main/DATOS}{link}.
\\ \\
Additionally, you can find a video of the data collection \href{https://github.com/LukasWolff2002/Lab_2_FM/blob/main/VIDEOS/VIDEO_CANAL.mp4}{here}.

\end{document}
